%!TEX TS-program = xelatex
%!TEX encoding = UTF-8 Unicode
%!TEX root = 2024-gs-adonis-template.tex
%
\documentclass{gs-adonis}
\usepackage[english,italian]{babel}
%-------------------------------------------------------------------------------
%--------------------------------------------------------------------- CUSTOMS -
%-------------------------------------------------------------------------------
% \usepackage{showframe}
% \renewcommand\ShowFrameLinethickness{0.15pt}
% \renewcommand*\ShowFrameColor{\color{red}}
% %
% \usepackage{tikz}
% \usetikzlibrary{
%   shapes,
%   through,
%   calc,
%   intersections,
%   backgrounds,
%   positioning,
%   decorations.text
%   }
% \usetikzlibrary{arrows.meta}
% \usetikzlibrary{bending}
\usepackage[
  small,
  labelfont=bf,
  up,
  textfont=it,
  up
  ]{caption}
%
\usepackage{paralist}
\usepackage{subfigure}
%\usepackage{glossaries}
%\input{includes/glossario.tex}
%\makeglossaries
\usepackage{csquotes}
\usepackage[style=ieee,backend=biber]{biblatex}
\bibliography{includes/bibliografia.bib}
\usepackage{enumitem}
\usepackage{adjustbox}
\usepackage{hhline}
\DeclareLabelname[movie]{
    \field{director}
    \field{producer}
  }
\usepackage{scrextend}
\usepackage{calc}
\usepackage{mwe}

\usepackage{todonotes}
\usepackage{hyperref}
%-------------------------------------------------------------------------------
%------------------------------------------------------------------------ MAIN -
%-------------------------------------------------------------------------------
\title{Autocostruzione di un interprete. Per quale musica?}
\subtitle{sottotitolo?}
\author{Alice Cortegiani \textsuperscript{1}}
% secondary details
%\affiliation{\textsuperscript{1} Spherical Technologies SRLS}
\correspondence{alicecortegiani@gmail.com}
\version{\today}
% headers
\runningauthor{Alice Cortegiani}
\runningtitle{Autocostruzione di un interprete. Per quale musica?}
%-------------------------------------------------------------------------------
%--------------------------------------------------------------------- COMANDI -
%-------------------------------------------------------------------------------
% \newcommand{\studi}{\emph{Sei studi di Agamotto sul Tempo}}
% \newcommand{\tempo}{\emph{Tempo}}
% \newcommand{\canto}{\emph{canto alla durata}}
%-------------------------------------------------------------------------------
%-------------------------------------------------------------------- ABSTRACT -
%-------------------------------------------------------------------------------
\abstract{%
  %\begin{addmargin*}[0pt]{-\marginparsep-\marginparwidth}
  \input{includes/abstract.txt}
  %\end{addmargin*}
}
%-------------------------------------------------------------------------------
%-------------------------------------------------------------------- DOCUMENT -
%-------------------------------------------------------------------------------
\begin{document}
\maketitle
%\input{includes/000-citazioni.tex}
%
%\clearpage
%-------------------------------------------------------------------------------
%--------------------------------------------------------------------- APPUNTI -
%-------------------------------------------------------------------------------
% \section*{APPUNTI}
% \begin{compactitem}
%   \item questione etica
%   \item diagrammi a blocchi
%   \item storia di una ricerca, fagioli, sogno, immagine
%   \item rappresentazione
%   \item feedback
%   \item interfaccia
%   \item tecnica
%   \item musica
%   \item invenzione
%   \item{bibliografia da compilare:}
%   \begin{compactitem}
%     \item Bergson
%     \item Barthes
%     \item Borges
%     \item Branchi
%     \item Cacciari
%     \item Candiani
%     \item Di Scipio
%     \item Galante
%     \item Guaccero
%     \item Handke
%     \item Ferraris
%     \item Lupone
%     \item Netti
%     \item Nono
%     \item Ronchi
%     \item Sartre
%   \end{compactitem}
% \end{compactitem}
%
%\clearpage
%-------------------------------------------------------------------------------
%--------------------------------------------------------------------- APPUNTI -
%-------------------------------------------------------------------------------
%\tableofcontents
%\clearpage
%-------------------------------------------------------------------------------
%-------------------------------------------------------------------- INCLUDES -
%-------------------------------------------------------------------------------
% \input{includes/000-introduzione.tex}
% \input{includes/100-ciclobase.tex}
% \input{includes/200-cad.tex}
% \input{includes/300-tempo.tex}

% \clearpage
%
% \tiny
% \twocolumn
% \printglossary[title={glossarietto}]
%
% \clearpage

%\normalsize
%\onecolumn

\section{Key words}

[massimo 5 key words]


\section{Descrizione del progetto di ricerca}% (massimo 2.000 parole / 2.000 words)}

\subsection{1. Descrizione del soggetto di ricerca (600 parole):}

\emph{a) descrivere l’ambito generale e lo stato dell’arte della pratica musicale in cui e attraverso cui si desidera svolgere il proprio progetto;}

La ricerca sul suono, nelle possibilità individuabili attraverso la ricerca
musicale, nell'articolazione delle peculiarità di cui dispone, determina
l'ambito generale in cui il progetto si inscrive tracciando percorsi unici e di
indipendenza dalla ricerca scientifica e universitaria. I luoghi della ricerca,
i soggetti coinvolti e gli oggetti individuati sono accessibili e attivi solo
in un processo musicale.

L'ambito generale in cui il progetto si inscrive è quello della musica di
ricerca che, con radici profonde nei piccoli laboratori sperimentali del
novecento, oggi è fortemente rappresentato da centri di ricerca storici e
nuovi laboratori indipendenti, la cui attività musicale si riversa nella
pratica, nella didattica e nella divulgazione musicale a piu livelli. Questo
livello di contro-reazione (feedback) è possibile solo in condivisione di un
processo musicale mediante l'ascolto. 

Il progetto si struttura attraverso le relazioni tra opera, strumento,
interprete mediante indagini di analisi e scrittura dove lo strumento definisce
la consapevolezza verso un pensiero creativo con la sua relativa prassi, non
incline a fini prestabiliti ma in continua tendenza verso presupposti utopici.
In ascolto.

Il progetto è inserito in una pratica quotidiana di ricerca presso laboratori
e centri specializzati.

Lo stato dell'arte della pratica musicale da cui il progetto è scalfito, pone,
ad un fare musicale attento,la necessità di un pensare creativo e analitico il
ruolo dell'interprete, poiché attraverso gli esiti dall'alta (de)formazione
artistica, facilmente si confonde la figura dell'esecutore con il ruolo
dell'interprete, precludendo così fondamentali contributi nel campo della
ricerca musicale, di un fare musicale condiviso, consumato nel fine
prestabilito dal dominio dell'intrattenimento.

% #### SUONO, RUMORE, TIMBRO
% Il suono, nella sua infinita varietà e complessità, rappresenta un universo
% ancora non completamente esplorato. La musica, in particolare, offre un terreno
% fertile per indagare le profonde connessioni tra suono, sensazione e significato.
% Questa tematica di ricerca si propone di inve- stigare le molteplici sfaccettature
% della creazione, manipolazione e percezione timbrica in contesti musicali.
% Il progetto di ricerca relativo a questa tematica può comprendere:
%
% • lo studio delle proprietà fisiche e psicoacustiche che definiscono il timbro
% di un determinato strumento, di una voce o di un suono complesso;
%
% • sperimentazione nella propria pratica compositiva o performativa di nuove tecnologie
% per la sintesi, il campionamento e la manipolazione del suono;
%
% • creazione di paesaggi sonori immersivi che esplorino le potenzialità del suono
% nello spazio e nella percezione umana;
%
% • uso innovativo nella propria pratica musicale di strumenti tradizionali attraverso
% tecniche estese.

\emph{b) formulare il problema e una o più domande di ricerca relative ad esso che possano guidare l’esplorazione dell’argomento.}

Nel campo di una formazione basata sul "repertorio", non vengono forniti gli strumenti
di ascolto e lettura della letteratura; la prassi musicale contemporanea decede dal
momento in cui si è inosservanti dei minimi termini per cui un musicista dovrebbe aver
cura e responsabilità; due esempi:
Si parla di timbro da nozioni di acustica senza l'approfondimento e la focalizzazione musicale.
Si parla di tecniche estese dello strumento quando l'uomo stesso non estende sé stesso
nell'esplorazione dello strumento.

\subsection{2. Metodi e processo di ricerca (600 parole):}

\emph{a) descrivere cosa si intende fare in termini pratici per indagare il proprio argomento di ricerca;}

\emph{b) indicare in che modo si prevede di coniugare le proprie capacità speculative e la propria pratica artistica in modo che diventino parte integrante del proprio metodo di ricerca.}

\subsection{3. Possibili risultati (300 parole):}

\emph{a) descrivere la forma che, al momento, il proprio lavoro finale di dottorato potrebbe assumere (tesi scritta, composizioni, performance, altri media e/o una combinazione di questi);}

Tesi, Opere, Concerto.

\emph{b) suggerire ulteriori modi di disseminazione e condivisione dei risultati della propria ricerca con le comunità artistiche e di ricerca, e con il pubblico in generale, durante e dopo gli studi di dottorato.}

concerto
seminari
articoli
concerti seminari
laboratorio

\subsection{4. Rilevanza per la conoscenza, comprensione e pratica musicale (500 parole):}

\emph{a) specificare in cosa consista l’originalità e la novità della propria prospettiva di ricerca;}

Focus interprete - responsabilità - cosapevolezza
nonché metodo di insegnamento volto ad un'attitudine consapevole, dello strumento e del
contributo che il musicista dona in merito a lettura del repertorio e relazione in ascolto
del contemporaneo.

\emph{b) descrivere dettagliatamente come il proprio progetto si relaziona alle diverse comunità di artiste/i e ricercatrici/ori e come i risultati della propria ricerca si potranno inserire negli ambiti di saperi e pratiche artistiche esistenti, in continuità o in contrasto con le conoscenze ereditate.}

\raggedright
\nocite{*}
%\bibliographystyle{unsrt}
\printbibliography

\end{document}
