%!TEX TS-program = xelatex
%!TEX encoding = UTF-8 Unicode
%!TEX root = 2024-gs-adonis-template.tex
%
\documentclass{gs-adonis}
\usepackage[english,italian]{babel}
%-------------------------------------------------------------------------------
%--------------------------------------------------------------------- CUSTOMS -
%-------------------------------------------------------------------------------
\usepackage[
  small,
  labelfont=bf,
  up,
  textfont=it,
  up
  ]{caption}
%
\usepackage{paralist}
\usepackage{subfigure}
\usepackage{csquotes}
\usepackage[style=ieee,backend=biber]{biblatex}
\bibliography{includes/bibliografia.bib}
\usepackage{enumitem}
\usepackage{adjustbox}
\usepackage{hhline}
\DeclareLabelname[movie]{
    \field{director}
    \field{producer}
  }
\usepackage{scrextend}
\usepackage{calc}
\usepackage{mwe}

\usepackage{todonotes}
\usepackage{hyperref}
%-------------------------------------------------------------------------------
%------------------------------------------------------------------------ MAIN -
%-------------------------------------------------------------------------------
\title{Autocostruzione di un interprete.\\
       Per quale musica?}
\subtitle{Progetto di \emph{Dottorato di Ricerca in Composizione e Performance musicale}}
\author{Alice Cortegiani \textsuperscript{1}}
% secondary details
%\affiliation{\textsuperscript{1} Spherical Technologies SRLS}
\correspondence{alicecortegiani@gmail.com}
\version{\today}
% headers
\runningauthor{Alice Cortegiani}
\runningtitle{Autocostruzione di un interprete. Per quale musica?}
%-------------------------------------------------------------------------------
%--------------------------------------------------------------------- COMANDI -
%-------------------------------------------------------------------------------
%\newcommand{\studi}{\emph{Sei studi di Agamotto sul Tempo}}
\newcommand{\tempo}{\emph{Tempo}}
\newcommand{\canto}{\emph{canto alla durata}}
%-------------------------------------------------------------------------------
%-------------------------------------------------------------------- ABSTRACT -
%-------------------------------------------------------------------------------
\abstract{%
  %\begin{addmargin*}[0pt]{-\marginparsep-\marginparwidth}
  \input{includes/abstract.txt}
  %\end{addmargin*}
}
%-------------------------------------------------------------------------------
%-------------------------------------------------------------------- DOCUMENT -
%-------------------------------------------------------------------------------
\begin{document}
\maketitle
%-------------------------------------------------------------------------------
%-------------------------------------------------------------------------------
%------------------------------------------------------------------ KEYWORDS ---
%-------------------------------------------------------------------------------
%-------------------------------------------------------------------------------
\section*{keywords}
clarinetto, interpretazione, timbro, elettroacustica, aumentazione.
%-------------------------------------------------------------------------------
%-------------------------------------------------------------------------------
%------------------------------------------------------ DESCRIZIONE PROGETTO ---
%-------------------------------------------------------------------------------
%-------------------------------------------------------------------------------
\section{descrizione del progetto di ricerca}% (massimo 2.000 parole / 2.000 words)}
%-------------------------------------------------------------------------------
%-------------------------------------------------------------------------------
%------------------------------------------------------ DESCRIZIONE SOGGETTO ---
%-------------------------------------------------------------------------------
%-------------------------------------------------------------------------------
\subsection{descrizione del soggetto di ricerca}% (600 parole):}
La ricerca sul suono, nelle possibilità individuabili attraverso la ricerca
artistica musicale, nell'articolazione delle peculiarità di cui essa dispone,
determina l'ambito generale in cui il progetto si inscrive tracciando percorsi
unici e di indipendenza dalla ricerca scientifica e universitaria. I luoghi
della ricerca, i soggetti coinvolti e gli oggetti individuati sono accessibili
e attivi al solo processo artistico musicale.

L'ambito generale in cui il progetto si inscrive è quello della musica di
ricerca che, con radici profonde nei piccoli laboratori sperimentali sorti nel
novecento, oggi è fortemente rappresentato da centri di ricerca storici e
nuovi laboratori indipendenti, la cui attività si riversa nella pratica, nella
didattica e nella divulgazione musicale a più livelli.

Nel contesto così descritto, parole come \emph{Suono, Rumore, Timbro},
%con indipendenza di indagine da altre discipline, nella ricerca musicale
acquisiscono strumenti unici di esperienza e conoscenza: ascolto, ambiente,
ambiente in ascolto (gli ascoltatori), in un livello di contro-reazione
(feedback) possibile solo in condivisione di un processo musicale.

Il progetto si struttura attraverso le relazioni che intercorrono tra opera,
strumento (il clarinetto) e interprete nel processo creativo di ricerca che
porta alla produzione di nuova musica. Mediante analisi del repertorio che
concede un'esplorazione radicale dello strumento, analisi dei segnali prodotti
dallo strumento con l'uso di tecnologie in grado di descriverne il
comportamento spaziale (che coincide con quello timbrico), una scrittura
dedicata all'esplorazione (dello strumento e della scrittura stessa), il
clarinetto si definisce a luogo di pensiero, di esperienza e consapevolezza,
verso un pensiero creativo che possa definirsi in prassi.
%in continua tendenza verso presupposti utopici, in ascolto.

Il progetto nasce da un'attività quotidiana di ricerca presso laboratori
e centri specializzati, in una pratica musicale attenta alle necessità di un
pensare creativo e analitico, dove il ruolo dell'interprete possa rinascere
dalle ceneri dell'intrattenimento cameristico, operistico e sinfonico.

\begin{figure}[t]
  \centering
  \includegraphics[width=\linewidth]{images/luigi-nono-massimo-cacciari.jpg}
  \captionsetup{width=.81\linewidth}
  \caption{Masimo Cacciari, Luigi Nono.}
  \label{cacciari}
\end{figure}

%\emph{b) formulare il problema e una o più domande di ricerca relative ad esso che possano guidare l’esplorazione dell’argomento.}

\begin{figure}[t]
  \centering
  \includegraphics[width=\linewidth]{images/panthera.pdf}
  \captionsetup{width=.81\linewidth}
  \caption{Interprete?}
  \label{alice}
\end{figure}

Massimo Cacciari nel descrivere il tema dell'ascolto in Luigi Nono
\cite{Cacciari1995} disegna lo sfondo di una problematica filosofica che assale
la cultura occidentale nelle relazioni tra scrittura, voce e ascolto, %Problemi di
%ordine generale che si riferiscono al significato culturale generale della
%nascita della scrittura alfabetica che in un lento processo vedono il dominio
%della \emph{visione} sull'\emph{ascolto}
in una

\begin{quote}
  progressiva desomatizzazione della voce. Una perdita dell'udire. L'udire
  non è più una funzione fondamentale del comprendersi. \cite{Cacciari1995}
\end{quote}

Egli sottolinea che in molte pratiche artistiche questo può non essere un
problema, ma non nella musica: la musica non può essere senza ascolto.
La perdita della memoria, dell'ascolto, della memoria dell'ascolto è fatale
per la musica. %Questo accade perché ad ogni ascolto muta il testo stesso.
%La musica non può esistere senza un ascolto vivente, attivo.

\emph{Che cos'è un interprete?}

In questo quadro di sensibilità musicale, l'interprete per primo, cercando di
comprendere, di capire che vorrebbe riuscire a comprendere di più ciò che c'è
\emph{«prima del primo suono e dopo l'ultimo suono»}, si fa testimone della
ricerca musicale nel luogo sonoro in cui opera in grado di
\emph{«articolare, accentuare, dare al canto»}, il testo musicale: a chiarire
anche la distanza dall'esecutore \emph{«l'interprete non legge»} il testo
musicale, \emph{«lo riattiva, lo da al canto»} \cite{Cacciari1995}.

\emph{Che cos'è il repertorio?}

Intraprendere il percorso di interprete della musica contemporanea di ricerca è
un atto che conduce inevitabilmente all'individuazione di problemi nel campo
della formazione accademica. Questa, incentrata sullo studio del repertorio
classico, attraverso gli esiti della de/formazione, confonde il ruolo
dell'esecutore con la figura dell'interprete, %consumato nel fine prestabilito
%dal dominio dell'intrattenimento,
%cristallizzando la sensibilità del musicista in una prassi parziale, ma resa
%assoluta, quindi distorta e
precludendo così fondamentali contributi nel campo della ricerca musicale, di un
fare musicale condiviso e contemporaneo.

% I programmi di studio dei corsi strumentali adottati dai conservatori di musica
% italiani favoriscono un percorso formativo fondato sul repertorio d'intrattenimento,

% La prassi del repertorio di musica
% contemporanea di ricerca costruisce e si costruisce attraverso lo strumento
% come strumento di pensiero, l'interprete è anello attivo della catena, generato
% e generatore.
%
% con conseguente esito di edificazione del musicista attraverso una parziale e distorta prassi, che rende totalizzante la visione limitata, ridotta.

% I programmi di studio dei corsi strumentali adottati dai conservatori di musica italiani favoriscono un percorso formativo
% fondato sul repertorio d'intrattenimento, cristallizzando la sensibilità del musicista in una prassi parziale resa assoluta,
% quindi distorta.
% Riconoscere la pluralità del repertorio è elemento essenziale per avere accesso, dalla teoria,
% alla complessità dei linguaggi musicali strutturati.
% La prassi del repertorio di musica contemporanea di ricerca costruisce e si costituisce nel processo;
% attraverso lo strumento acustico, strumento di pensiero, l'interprete è anello attivo della catena, generato e generatore.
%
% Nel dominio dell'intrattenimento la chiave di lettura tende, sovente, erroneamente a sottovalutare aspetti fondamentali di apporto storico-sociale
% che hanno nutrito la creazione musicale, favorendo ambizione e gratificazione nel risultato alla riproduzione del testo scritto.
% Reperire, trovare, partecipare al processo creativo di musica contemporanea di ricerca comporta un costante confronto:
% con le domande fondamentali di cui si nutre la prassi interpretativa, con gli strumenti da costruire per avere accesso al dialogo con
% interlocutori dediti ad pensiero in ascolto, volti alla creazione e quindi nel reperire opere di ricerca dal pensiero musicale.

\emph{Che cos'è contemporaneo?}

\begin{quote}
  Contemporaneo è colui che riceve in pieno viso il fascio di tenebra che
  proviene dal suo tempo. \cite{agamben2008che}
\end{quote}

La contemporaneità è quindi un momento mobile del tempo che identifica la
facoltà di osservare l’oscurità del tempo specifico, quella relazione col
tempo che aderisce a esso attraverso una sfasatura e un anacronismo che ci
permette di valutare, vedere ed analizzare, alla dovuta distanza.
\emph{Distanza da cosa?}

Fare repertorio contemporaneo, nella contemporaneità, nell’espressione del suo
senso più completo è imparare ad ascoltare: \emph{Suono, Rumore, Timbro} e Silenzio.

\emph{Che cos'è la ricerca musicale contemporanea e come può formarsi l'interprete contemporaneo?}
%-------------------------------------------------------------------------------
%-------------------------------------------------------------------------------
%--------------------------------------------------------- METODI E PROCESSO ---
%-------------------------------------------------------------------------------
%-------------------------------------------------------------------------------
\subsection{metodi e processo di ricerca}% (600 parole):}

Nella teoria musicale odierna, la semplicistica nozione di \emph{timbro} come
una delle quattro caratteristiche dei suoni percepite dall'uomo, risulta essere
estremamente riduzionista e privativa. Riduzionista, perché slega le relazioni
tra le quattro caratteristiche con principi di indipendenza che non hanno
riscontri fisici (un'altezza in un registro grave di un clarinetto può produrre
diversi timbri, alcuni dei quali molto lontani dall'idea di timbro come
caratteristica del suono di un clarinetto). Privativa perché preferendo una
teoria acustica ad una musicale, l'impianto è privato di tutto ciò che è stato
esplorato nel novecento dall'introduzione della teoria dei timbri (con origine
in \emph{Farben}, Schoenberg), non procedendo alla speculazione su quelle
scoperte, attorno a quelle teorie, destinandole all'oblio.

Così lo studio di \emph{Luz}\footnote{%
Domenico Guaccero, \emph{Luz (da Descrizione del corpo) per strumento grave}, 1973
} apre lo spiraglio da cui spiare un mondo nuovo: \emph{cosa accadrebbe se lo
studio dello strumento non fosse logocentrico (le altezze e le loro relazioni)
ma timbrico, in una teoria del timbro che produca altezze, durate ed intensità
inalienabili dalla scelta timbrica operata?}

%\emph{a) descrivere cosa si intende fare in termini pratici per indagare il proprio argomento di ricerca;}
%Il percorso di ricerca parte dall'analisi musicale di luz (già avviata con quattro interpretazioni in concerto)
%e giunge alla catalogazione sistematica delle possibilità timbriche del clarinetto
%contrabbasso nelle due tipologie (metallo e legno) sotto la guida di …

Il metodo di indagine vede quindi come primo passo l'analisi del repertorio
specifico, \emph{timbrico}, in cui \emph{Luz}, per l'imponenza teorica, si
dispone a manuale operativo. Guaccero è stato musicista dalla vocazione alla
ricerca artistica (e) musicale, ha osservato con vivacità ed acutezze plurime
tessiture di contemporaneità alimentata dal fare musicale e azione sociale.
% \emph{Luz}, dalle emersioni del corpus di opere, si pone in continuità con
% il Contemporaneo in quanto illumina dal fascio di tenebre da cui è forgiata.
In \emph{Luz} articola melodie di \emph{Timbri} di 24 differenti tipologie, con
una scrittura che sintetizza oltre un decennio di ricerca grafica. Espone la
ricerca timbrica alla relazione con il silenzio, mediante un artificio
compositivo geniale: l'introduzione di un silenzio \emph{udibile animato}.

Nel metodo di studio introdotto analizzando il brano, timbro, silenzio e grafia
musicale fondono il nucleo centrale della spirale speculativa di esperienza e
conoscenza e alimentano l'individuazione di un nuovo grado di interpretazione,
di un nuovo interprete.

Il clarinetto contrabbasso è il più grande dei luoghi clarinettistici da esplorare,
in due versioni, due città diverse \cite{netti23}: legno e metallo. Lo studio
comparato sulla stessa opera può far emergere considerazioni importanti per la
conoscenza dello strumento.

Da \emph{Luz} in poi, lo studio di un brano di repertorio \emph{timbrico}
avviene all'interno di una struttura microfonica composta di quattro microfoni
omni-direzionali disposti a tetraedro denominata \emph{TETRAREC} e ideata da
Giuseppe Silvi\footnote{%
  Docente di Elettroacustica del Conservatorio di Bari; fondatore del LEAP -
  Laboratorio di ElettroAcustica Permanente di Roma.
  \url{l-e-a-p.github.io}
} per poter osservare e conservare le evoluzioni spaziali che corrispondono agli
incedere timbrici.

Lo studio dello spazio interno ed esterno dello strumento, l'osservazione di
questo mediante apparati tecnologici, in una ricerca musicale, si deve
necessariamente ed immediatamente confrontare con l'ambiente del concerto e
l'ambiente in ascolto. Per questo motivo alla tecnica microfonica è associata
una tecnica di diffusione tridimensionale basata su sistema S.T.ONE\footnote{%
  \url{l-e-a-p.github.io/giuseppe/stone}
}

%L'opera ha dato impulso al progetto di ricerca. Dal gennaio del 2023 è stata
%ascoltata in quattro interpretazioni in concerto, complementari alle indagini
%di laboratorio. Allo stato attuale giunge alla catalogazione sistematica delle
%possibilità timbriche del clarinetto contrabbasso nelle due tipologie, metallo
%e legno attraverso tecnologie e strumenti di ascolto [\ldots]

\begin{figure}[t]
  \centering
  \includegraphics[width=\linewidth]{images/IMG_F5A47D566D6E-31.jpeg}
  \captionsetup{width=.81\linewidth}
  \caption{Sistema Bo.Si.}
  \label{bosi}
\end{figure}

Il percorso già attuato vedrebbe un primo approfondimento con Giorgio Netti,
nell'ipotesi di un laboratorio condiviso basato sul suo lavoro:
\emph{Che cos'è uno strumento?}\cite{netti23}, reso specifico da una catalogazione
a quattro mani sul clarinetto contrabbasso, strumento-luogo, in un primo passo
di scrittura di esplorazione, dello strumento e della scrittura stessa.

\begin{quote}
  Chi ha esperienza nel campo dell’esplorazione (non specificatamente
  strumentale) sa che molto spesso la quantità d’informazione relativa ad un
  argomento è talmente ricca da diventare essa stessa un problema. Dove tutto è
  differentemente importante nulla è più importante e si rimane abbandonati
  sulla riva senza saper che fare: dunque, farsi ammaliare dal suono-sirena sì
  ma, prendendo esempio da Ulisse, nume tutelare di tutti gli esploratori,
  farsi ammaliare legati [\ldots] esplorare gli strumenti come se fossero città
  antiche: ognuno di loro con un suo centro storico costituito da tutto ciò a
  cui scolasticamente associamo a quello strumento ed una periferia,
  progressivamente meno definita, fatta da ciò che è incerto, meno omogeneo,
  nascosto. \cite{netti23}
\end{quote}

\begin{figure}[t]
  \centering
  \includegraphics[width=\linewidth]{images/CAD-IV-GRAFO-ANNOTATO.pdf}
  \captionsetup{width=.81\linewidth}
  \caption{Grafo}
  \label{grafo}
\end{figure}

%Esplorando il clarinetto contrabbasso, luogo di pensiero, in maniera radicale,
%sferica, è subentrata la necessità di estendere la pratica corporea ad una
%conoscenza più stratificata, approfondita e lontana dai filtri di quanto ad oggi
%risulta evinto.

% In un percorso verso la contestualizzazione, l'integrazione e il superamento
% dell’interpretazione abituale dei vari elementi caratterizzanti dello strumento
%
% \begin{quote}
% \end{quote}

L'approccio metodologico è quello di esplorare lo strumento dalla prospettiva
particolare di inzio, esordio, come se non fosse mai stato ascoltato,
predisporre zone periferiche dello stesso per una prima mappatura che tracci
quelle particolarità,


%\begin{quote}
  %criticità, si relazionino all'intera storia dello strumento o piuttosto
  %come è ascoltata o pensata a partire da quella caratteristica, l'intera
  %storia dello strumento assume una prospettiva differente che in dialogo con
  %le precedenti contribuisce ad estenderne i confini, e quindi, il senso.
%\end{quote}

%verso l'organizzazione di un \emph{Catalogo} che emerga dall' \emph{andare incontro} allo strumento.

Il passo successivo alla conoscenza è la rimessa in contro-reazione del nuovo
tutto con nuovi esemplari: l'affondo epistemologico produce nuove ontologie:
dallo strumento esplorato si procede per l'estensione, la proiezione di questo
verso problemi aumentati.

Innestato nel ciclo \emph{canto alla durata} di Giuseppe Silvi
\cite{gs:cad, gs:artQ13}, per quattro strumenti aumentati, è previsto un lavoro
di sviluppo di una sordina aumentata per clarinetto contrabbasso (una per ogni
versione, legno e metallo), sulla linea di ricerca già intrapresa per la sordina
Bo.Si. per Euphonium (\ref{bosi}).

Il progetto prevede quindi una serrata metodologica attorno a problemi di
scrittura e creatività mediante la realizzazione di opere nuove ed originali,
dedicate al progetto.

La permanenza all'estero può avvenire, in funzione dell'orientamento e della
gestione delle risorse, sia in Francia, per sviluppare con Selmer alcune degli
elementi tencici per la sordina per il clarinetto contrabbasso in legno; sia
in università in cui la ricerca musicale può essere accolta affiancando
rilevazioni tecniche, per esempio in luoghi dove è accessibile una camera anecoica.
%-------------------------------------------------------------------------------
%-------------------------------------------------------------------------------
%------------------------------------------------------- POSSIBILI RISULTATI ---
%-------------------------------------------------------------------------------
%-------------------------------------------------------------------------------
\subsection{Possibili risultati}% (300 parole):}

%\emph{a) descrivere la forma che, al momento, il proprio lavoro finale di dottorato potrebbe assumere (tesi scritta, composizioni, performance, altri media e/o una combinazione di questi);}

Il progetto di dottorato proposto prevede diversi esiti:

\begin{description}
  \item[Pubblicazione analisi Luz] L'analisi del brano \emph{Luz} può essere,
    sulla base del lavoro d'interpretazione già svolto, la prima pubblicazione
    istituzionale verso un lavoro più corposo e complesso di analisi. Il 2027
    sarà il centenario dalla nascita di Domenico Guaccero e il lavoro
    sull'autore potrebbe portare alla partecipazione a conferenze e convegni
    sulla figura del Compositore di Palo del Colle.
  \item[Tesi di Dottorato] \emph{Autocostruzione di un Interprete} potrà essere il
    progetto di Tesi in cui raccogliere gli aspetti teorici emergenti
    dall'affinamento del metodo di ricerca, gli approfondimenti analitici, sia
    in termini di trattamento del segnale, sia in termini musicologici e il
    raffinamento di processi didattici derivanti dal metodo stesso;
  \item[Composizione] originale per Clarinetto Contrabbasso ed elettronica in
    collaborazione con il Compositore Michelangelo Lupone e supportata dal
    CRM - Centro Ricerche Musicali;
  \item[Composizione] originale per Clarinetto Contrabbasso Aumentato e
    \emph{Tempo}\footnote{%
  \emph{Tempo - Timpani ElectroMagnetic Pulse Oscillation}, timpano
  elettromagnetico progettatto da Giuseppe Silvi.
  };
  \item[Concerto] Il concerto finale del percorso vuole collegare in un arco di
    ascolto il percorso di ricerca e il percorso riflessivo attorno al concetto di
    esplorazione timbrica dello strumento clarinetto: \emph{Luz}, per strumento
    grave e silenzio animato; opera originale per clarinetto contrabbasso ed
    elettronica; opera originale per clarinetto contrabbasso aumentato e timpano
    elettromagnetico.
\end{description}

%\emph{b) suggerire ulteriori modi di disseminazione e condivisione dei risultati della propria ricerca con le comunità artistiche e di ricerca, e con il pubblico in generale, durante e dopo gli studi di dottorato.}

La motivazione che ha fatto da propulsore all'esperienza finora accordata,
all'eplorazione che avanza dentro e fuori lo strumento, è il desiderio di
ricerca e la sua naturale restituzione in contesti di attenta condivisione. Per
questo motivo, dietro ad un progetto come questo si cela la speranza, semplice,
che un giorno tutto questo lavoro possa essere d'interesse e d'ausilio per una
didattica rinnovata dello strumento e della musica ad esso possibile.

A tal fine si immaginano pubblicazioni analitico-interpretative, teoriche
laboratori e seminari concertati, in cui il dialogo attorno alle
tematiche di \emph{Suono, Rumore, Timbro} e Silenzio possano avvalersi dell'ascolto
attivo e partecipato di esemplari musicali. Costruire un metodo di studio, per
costruire un nuovo modo di fare ascolto e pubblico.

\subsection{Rilevanza per la conoscenza, comprensione e pratica musicale}
%(500 parole):}
Il concetto di «autocostruzione» mutuato da Enzo Mari \cite{mari2002} è più una
sintesi che uno stimolo letterario: la garanzia

\begin{quote}
  di "contrabbandare", dentro le maglie delle […] realizzazioni, momenti di
  ricerca e contributi per lo stimolo a uscire dai condizionamenti ideologici,
  normativi, di comportamento e di gusto
\end{quote}

ceneri dell'imperialismo culturale.

La partecipazione attiva a contesti di ricerca musicale già in essere, seppur
sviluppati in forma autonoma non istituzionalizzata, è garante di un'autonomia
di lavoro, di organizzazione e collaborazione che, con il compimento di un
percorso di Dottorato di Ricerca Musicale in Conservatorio, potrebbe riversari,
strutturata e ampliata, nella didattica e nella professione. La collaborazione
quotidiana con gruppi di ricerca e interpretazione, l'uso di tecnologie di
analisi dei segnali e di strumenti di invenzione, ampliano la conoscenza
delle possibilità creative allo strumento e a loro volta beneficerebbero
dell'apporto istituzionale, di relazioni e condivisibilità.

\begin{quote}
  …andare alla radice del suono, come fatto fisico e da qui come fatto musicale
  […] riproporre, ma allargandone permanentemente i confini, il rapporto tra
  tecnologia e composizione.
\end{quote}

%Lo stato dell'arte in cui si innesta il progetto di ricerca è teso da un continuo
%approfondimento della capacità di analisi nel ruolo dell'interprete.
%Osservare le tecnologie: strumenti musicali: strumenti di pensiero, rende possibile
%la visione di tracce di evoluzione del pensiero musicale sottili, rilevanti:
%un clarinetto evoluto apre per Mozart un varco ulteriore al proprio
%sistema di linguaggio.
%Interpretare un'opera di Domenico Guaccero, ri-apre la necessità di progredire
%nell'analisi verso un pensare oggi le tecnologie.

%\begin{quote}
%  \ldots un pensiero nuovo, con la sua forza dirompente, ha dato il primo impulso,
%  cioè dove, prima ancora di ogni formulabile etica, la spinta morale è stata
%  abbastanza grande da concepire e progettare una nuova possibile etica.
%\end{quote}


%\begin{quote}
%  capacità di cambiare l'organizzazione dei nostri pensieri che ci permette
%  salti in avanti
%\end{quote}

%La coniugazione del rigore scientifico alla creatività del pensiero analogico
%suggerisce preziose insenature di esplorazione.

%\emph{Luz, da Descrizione del corpo} di Domenico Guaccero è una delle opere del
%corpus di oggetto di indagine.

%In una pratica laboratoriale quotidiana condivisa, non possiamo che
%\emph{saltare} interrogando l'analisi del concetto di timbro attraverso
%l'interpretazione di \emph{Luz}.

%Guaccero crea un \emph{luogo di ascolto} per strumento grave, l'opera rompe
%gli argini saltando nell'esperienza del suono verso la
%\emph{soglia dell'udibilità} con l'interpretazione di 24 tipologie di timbri
%differenti, accompagnati da un \emph{silenzio animato}.

%Non è sufficiente argomentare il timbro come il parametro di riconoscimento di
%uno strumento come il clarinetto.

%Il sapiente ascolto della tecnologia manipola l'esperienza del silenzio,
%evocando la soglia dell'arte attraverso un coro animato messo in vibrazione da
%[... :)]

% Con autocostruzione di un interprete si intende la sua realizzazione mediante
% «assemblaggi» di opere, strumenti e scritture come «tavole grezze e chiodi»,
% una forma di hacking del ruolo interpretativo e del suo palco, nel rito del
% concerto che indichi la soglia per l'ingresso di nuove cose nel mondo:
% «… perché ognuno possa porsi di fronte alla produzione
% attuale con capacità critica.» \cite{mari2002}

%\emph{b) descrivere dettagliatamente come il proprio progetto si relaziona alle diverse comunità di artiste/i e ricercatrici/ori e come i risultati della propria ricerca si potranno inserire negli ambiti di saperi e pratiche artistiche esistenti, in continuità o in contrasto con le conoscenze ereditate.}

Attualmente ci sono in essere:

\begin{description}
  \item[LEAP - Laboratorio ElettroAcustico Permanente, Roma] Dal Dicembre 2020
  appare sulle mappe di Roma, a due passi da Villa Lazzaroni, quartiere Appio
  Latino. Nonostante la geolocalizzazione, che dopotutto può essere solo
  temporanea, il Laboratorio punta ad essere Permanente in quanto oggetto sociale,
  nel luogo fisico delle persone che lo animano. ElettroAcustico, perchè nella
  storia della ricerca musicale romana, la necessità di trovare nell'ascolto
  il luogo d'unione tra tecnologia e strumento acustico. LEAP, salto, quello che
  facciamo con il pensiero inseguendo l'intuizione.

  %Il LEAP è il luogo eterotopico per la condivisione della ricerca musicale.
  %Nella dimensione di bottega, che ne alimenta l'attività quotidiana, la tecnologia
  %è organica alla musica e si riversa con strumenti nuovi, tecnici e di pensiero,
  %nelle possibilità creative del laboratorio. [LINK SITO]
  \item[LAZZARO] Prende forma tra i componenti del LEAP dalla necessità di una
  pratica musicale condivisa che medi il conosciuto e il possibile: è l'organismo
  del Laboratorio che articola i \emph{perché} che alimentano la ricerca verso
  i \emph{come} di condivisione con la società. È esercizio di prassi, interpretazione
  ed esplorazione in relazione con le due principali tecnologie della musica: lo
  strumento e la scrittura: un ponte tra le due isole in continua osservazione
  delle singole geografie. LAZZARO è un dispositivo retroattivo, lo specchio
  attraverso cui la sala da concerto è inizio vitale di ogni opera, la datazione
  istantanea di un fare musicale che articola il processo di ricerca e non il
  raggiungimento di un fine prestabilito.
  \item[Centro Ricerche Musicali CRM] Associazione no profit fondata a roma
  nel 1988 dai compositori Laura Bianchini e Michelangelo Lupone, per promuovere
  la ricerca musicale nei suoi aspetti estetici, analitici, musicologici e scientifici.
  Il CRM, nella sua produzione musicale, nell'attività concertistica in Italia e
  all'estero, progetta e sviluppa sistemi hardware e software mirati; sperimenta
  sistemi di composizione e algoritmi che permettono una costante interazione tra
  i linguaggi musicali, il pensiero scientifico e le risorse tecnologiche.
  %Dai laboratori del CRM sono usciti complessi sistemi digitali per la sintesi
  %e l'elaborazione del suono in tempo reale (computer \emph{esperti}) per la
  %composizione musicale, per la progettazione di spazi d'ascolto, per lo studio
  %di modelli fisici finalizzati allo sviluppo di strumenti musicali virtuali.
  %\item[Festival ArteScienza] manifestazione che presenta, in forma di spettacolo,
  %le più avanzate ricerche in ambito acustico, psicoacustico e scientifico-tecnologico,
  %propondendo soluzioni alternative ai modi tradizionali di fruizione e della composizione.
  \item[Inter Arts Center IAC / Lund University, Malmö] Istituto dedicato alla ricerca
  artistica e ricerca in educazione musicale, caratterizzato da una forte impronta
  di collaborazione in sinergia di partner accademici e artistici.
  Si propone come eventuale indirizzo per il periodo di ricerca all'estero.


\end{description}


Tutto ciò porterebbe ad un rapido capovolgimento di tendenza del potenziale
didattico del Conservatorio che nel giro di un ciclo di Dottorato disporrebbe
di uno strumento didattico nuovo, di uno strumento di pensiero per una nuova
didattica dello strumento.

% Allo scopo di favorire la creazione di un ambiente di ricerca dinamico e collaborativo, le attivi- tà didattiche, formative e artistiche sono organizzate in un unico curriculum ma avranno luo- go nelle quattro città sedi dei Conservatori promotori (Ferrara, Pescara, Trieste, Udine)

% L'attuale carriera di alta formazione artistica per uno studente di strumento,
% nel caso specifico il clarinetto, preclude possibilità di occupazione e
% creatività all'interno del panorama della ricerca musicale contemporanea. Ciò
% è fondamentalmente dovuto all'assenza della ricerca nei conservatori. Finora.
% Con l'introduzione dei Dottorati di ricerca si prospetta un cambio di
% prospettive nei confronti di una realtà artistica che fuori dalle istituzioni
% può vantare quasi un secolo di storia, esperienza e coscienza storica. Il
% progetto presentato si espone a giunizone di questo strappo proponendo
% metodologie e pratiche già in atto nelle collaborazioni con centri di ricerca e
% laboratori indipendenti. Il focus sulla figura dell'interprete, la riscrittura
% del suo futuro mediante repertorio “di responsabilità” e la relazione con
% la scrittura contemporanea sono ponti che possono collegare la tradizionale
% didattica in Conservatorio con le nuove possibilità delle relazioni
% contemporanee.

\clearpage
\raggedright
\nocite{*}
%\bibliographystyle{unsrt}
\printbibliography

\end{document}
